\subsection{K leading digits of n!}
A similar idea can be used to calculate the first digits of exponentiation.

$$ \log _{10} n! = \log _{10} (1 \times 2 \times 3 \times ... \times n) = \log _{b} 1 + \log _{10} 2 + \log _{10} 3 + ... + \log _{10} n $$

Decimal part:
$$ q = \log _{10} n! - (int) \log_{10} n! $$

Leading digits:
$$ b = pow(10, q) $$
\begin{lstlisting}[style=c++, numbers=none]
// Shift decimal point k-1 times
for ( int i = 0; i < k - 1; i++ ) {
    b *= 10;
}
\end{lstlisting}
$$ leadingdigits = \lfloor b \rfloor $$
